\documentclass[12pt, a4paper, czech]{article}
\usepackage[left=2cm,text={17cm, 24cm}, top=3cm]{geometry}
\usepackage[IL2]{fontenc}
\usepackage[utf8]{inputenc}
\usepackage[czech]{babel}
\usepackage{amsmath}
\usepackage{amssymb}
\usepackage{url}
\DeclareUrlCommand\url{\def\UrlLeft{<}\def\UrlRight{>} \urlstyle{tt}}


\begin{document}
	%%-----------------Title------------------------------------------
	\begin{center}
		\Huge
		\textsc{\Huge{Vysoké učení technické v~Brně}\\
			\huge{Fakulta informačních technologií}}\\
		\vspace{\stretch{0.382}}
		\Large{Typografie a publikování - 4. projekt}\\
		\Huge{Citace}\vspace{\stretch{0.618}}
	\end{center}
	\thispagestyle{empty}
	\normalsize{\today \hfill Jozef Urbanovský}
	\newpage
	
	\pagenumbering{arabic}
	\setcounter{page}{1}
	
	\section{Typografia}
	Slovo typografia je odvodené od slov \textit{type} a \textit{graph}, prenesene grafika textu.
	Pre naše účely zjednodušene tieto základy typografie znamenajú súhrn pravidiel, ktoré 
	podporujú zrozumiteľnosť a čitateľnosť textu. \cite{typo}	
	Typografia je v slovníkov vysvetľovaná ako kníhtlač, odbor, ktorý zahŕňa sadzbu, 
	tlač, výtvarné a technické riešenie dokumentu. 
	V dnešnej dobe je bežné realizovať typografiu a jej odvetvia
	na osobných počítačoch, za pomoci typografických systémov známych ako 
	DTP systémy. \cite{cerny} \par
	
	 
	Článkov o typografii môžeme nájsť na internete nespočtne mnoho, ale nemožno sa však
	pri výzkume opierať o cudzojazyčné zdroje, nakoľko každý jazyk má svoje typografické 
	zákonitosti. Pri zameraní sa len na české premene venujúce typografii, zistíme, že kvantita 
	prevyšuje kvalitu.  \cite{fiala} 
	\par
	
	\LaTeX\ je rozsiahla zbierka značkovacích príkazov používaných silným sádzacím 
	nástrojom \TeX, ktorý sa používa na širokú škálu dokumentov od vedeckých článkov 
	až po komplexné knihy. \cite{kopka} 
	\TeX (vyslovuje sa „tech“) je počítačový typografický systém, ktorý začal v roku 1977 vyvíjať
	americký informatik Donald Knuth. Hlavným Knuthovým impulzom pre tento počin bola predovšetkým
	zhoršujúca sa typografická kvalita postupne vydávaných častí jeho knihy \textit{The Art of
	Computer Programming} v kombinácii s potenciálom, ktorý videl v 
	počítačovej typografii. \cite{kosto} 
	Základný princíp \LaTeX u spočíva v tom, že autor by nemal byť neustále obťažovaný formátovaním.
	Napríklad, ak píše nadpis sekcie nemusí sa starať o tom, aby bol text zvýraznený, 
	ale stačí špecifikovať, že sa jedná o nadpis. Jeho štýl si stačí raz definovať 
	ako bude vyzerať v cieľovom dokumente a je tak možné zachovať konzistenciu.\cite{kottw}
	\LaTeX\ nemá konkurenciu v prípadne písania rozsiahlych textov, približne od 50 strán, 
	prípadne aj menej ak sa jedná o text obsahujúci množstvo vzorcov, tabuliek a obrázkov. 
	Je štandardom v drvivej väčšine odborných nakladateľstiev ako napríklad Elsevier, 
	Cambridge University Press, AIP a iné. \cite{groger} 
	\par
	
	Jadro systému tvorí prekladač jazyka \TeX\ spoločne s nadstavbou \LaTeX, ktorého výstupom
	je textový súbor, do ktorého sa zapisuje cieľový text spolu s príkazmi, ktoré určujú
	ako má byť text vysádzaný. Výstupom prekladača je súbor s vysádzaným textom, ktorý 
	je vo formáte nezávislom na zariadení - \texttt{*.dvi}. \cite{rybic}
	\par	

	Väčšina bežných textov je členená do hierarchickej štruktúry kapitol, kapitoliek a iných 
	menších častí. Bývajú očíslované a nadpisy zhodnej úrovne vyzerajú identicky. 
	\LaTeX\ má pre tento účel sadu príkazov, ako napríklad 
	\textit{\textbackslash chapter, \textbackslash section} a iné. \cite{satra} 
	\par
	
	Na záver je dobré pripomenúť, aká prínosná pre vás môže byť typografia, ako dokazuje článok \cite{wong}, ktorý poukazuje na výber správneho štýlu písma pri zasielaní žiadosti o prácu.
	
	
	
	
	

	
	
	
	
		
	

	

	
	
	\newpage
	\bibliographystyle{czplain}
	\renewcommand{\refname}{Literatúra}
	\bibliography{liter}
	
\end{document}		