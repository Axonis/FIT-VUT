%%-----------------Document Type----------------------------------
\documentclass[a4paper,11pt]{article}

%%-----------------Packages---------------------------------------
\usepackage[utf8]{inputenc}
\usepackage[czech]{babel}
\usepackage[IL2]{fontenc}
\usepackage{amsmath}
\usepackage{amsthm}
\usepackage{times}
\usepackage[left=1.5cm, text={18cm,25cm}, top=2.5cm]{geometry}

%%-----------------User theorems----------------------------------
\newtheorem{definice}[subsection]{Definice}
\newtheorem{algoritmus}[subsection]{Algoritmus}
\newtheorem{veta}{Věta}


\begin{document}
	
%%-----------------Title------------------------------------------
	\begin{center}
		\Huge
		\textsc{Fakulta informačních technologií\\
			Vysoké učení technické v~Brně}\\
		\vspace{\stretch{0.382}}
		\Large{Typografie a publikování - 2. projekt\\
			Sazba dokumentů a matematických výrazů}\vspace{\stretch{0.618}}
	\end{center}

	\thispagestyle{empty}
	\normalsize{2017 \hfill Jozef Urbanovský}
	\newpage
	\pagenumbering{arabic}
	\setcounter{page}{1}
	\twocolumn
	
%%-----------------Prolog section-----------------------------------	
	\section*{Úvod}
	V~této úloze si vyzkoušíme sazbu titulní strany, matematických vzorců, 
	prostředí a dalších textových struktur obvyklých pro technicky 
	zaměřené texty, například rovnice (\ref{rov1}) nebo definice \ref{def1.1} 
	na straně \pageref{def1.1}.\par
	
	Na titulní straně je využito sázení nadpisu podle optického středu 
	s~využitím zlatého řezu. Tento postup byl probírán na přednášce.\par

%%-----------------Math section------------------------------------	
	\section {Matematický text}
	Nejprve se podíváme na sázení matematických symbolů a~výrazů v plynulém textu. 
	Pro množinu $V$ označuje\linebreak card($V$) kardinalitu $V$.
	Pro množinu \emph{V} reprezentuje $V^*$ volný monoid 
	generovaný množinou $V$ s~operací konkatenace.
	Prvek identity ve volném monoidu $V^*$ značíme symbolem $\varepsilon$.
	Nechť $V^+=V^*-\{\varepsilon\}$ Algebraicky je tedy $V^+$ volná 
	pologrupa generovaná množinou $V$ s~operací konkatenace.
	Konečnou neprázdnou množinu $V$ nazvěme $abeceda$.
	Pro $\omega\in V^*$ označuje $|\omega|$ délku řetězce $\omega$. 
	Pro $W \subseteq V$ označuje occur($\omega,W$) 
	počet výskytů symbolů z~$W$ v~řetězci $\omega$ a sym($\omega,i$) 
	určuje $i$-tý symbol řetězce $\omega$;~například sym$(abcd,3)=c$.\par
	
	Nyní zkusíme sazbu definic a vět s~využitím balíku \verb$amsthm$.\par

%%-----------------Grammars definition-----------------------------	
	\begin{definice}
		\label{def1.1} Bezkontextová gramatika \textup{je čtveřice $G=(V,T,P,S)$, 
			kde $V$ je totální abeceda, $T \subseteq V$ je abeceda terminálů, 
			$S\in (V-T)$ je startující symbol a~$P$ je konečná množina 
			\emph{pravidel} tvaru $q\colon A\rightarrow \alpha$, 
			kde $A\in (V-T)$, $\alpha\in V^*$ a $q$ je návěští tohoto pravidla. 
			Nechť $N=V-T$ značí abecedu neterminálů.
			Pokud $q\colon A\rightarrow\alpha\in P$, $\gamma$, 
			$\delta\in V^*$, $G$ provádí derivační krok z~$\gamma A\delta$ do $\gamma\alpha\delta$ podle pravidla $q\colon A\rightarrow~\alpha$, 
			symbolicky píšeme $\gamma A\delta\Rightarrow\gamma\alpha\delta\;
			[q\colon A\rightarrow\alpha]$ nebo zjednodušeně $\gamma A\delta\Rightarrow\gamma\alpha\delta$. Standardním způsobem definujeme  $\Rightarrow^m$, kde $m\geq 0$. Dále definujeme tranzitivní uzávěr 
			$\Rightarrow^+$ a tranzitivně-reflexivní uzávěr $\Rightarrow^*$.}\par
	\end{definice}
	
	\textup{Algoritmus můžeme uvádět podobně jako definice textově, 
		nebo využít pseudokódu vysázeného ve vhodném prostředí (například} \verb$algorithm2e$\textup{).}
	
	\begin{algoritmus}
		Algoritmus pro ověření bezkontextovosti gramatiky. 
		Mějme gramatiku $G = (N, T, P, S)$.
		
		\begin{enumerate}
			\item \label{item1}Pro každé pravidlo $p\in P$ proveď test, 
			zda $p$ na levé straně obsahuje právě jeden symbol z~$N$.
			\item Pokud všechna pravidla splňují podmínku z~kroku \ref{item1}, 
			tak je gramatika $G$ bezkontextová.
		\end{enumerate}
	
	\end{algoritmus}
	
	\begin{definice}
		Jazyk \textup{definovaný gramatikou~$G$ definujeme 
			jako $L(G)=\{w\in T^* \mid S\Rightarrow^* w\}$.}
	\end{definice}
	
	\setcounter{subsection}{0}
	\subsection{Podsekce obsahující větu}
	
	\setcounter{subsection}{3}
	\begin{definice}
		{\textup{Nechť $L$ je libovolný jazyk. $L$~je} bezkontextový jazyk, 
			\textup{když a jen když $L=L(G)$, kde $G$ je libovolná bezkontextová gramatika.}}
	\end{definice}
	
	\begin{definice}
		\textup{Množinu $\mathcal{L}_{CF}=\{L\!\mid\! L$ je bezkontextový jazyk\}
			nazýváme \emph{třídou bezkontextových jazyků.}}
	\end{definice}
	
	\begin{veta}
		\label{veta1}
		Nechť $L_{abc}=\{a^nb^nc^n|n\geq 0\}$. Platí, že $L_{abc}\not\in\mathcal{L}_{CF}$.
	\end{veta}
	
	\begin{proof}
		Důkaz se provede pomocí Pumping lemma pro bezkontextové jazyky, kdy ukážeme, 
		že není možné, aby platilo, což bude implikovat pravdivost věty \ref{veta1}.
	\end{proof}

%%-----------------Equations & link section------------------------	
	\section{Rovnice a odkazy}
	
	Složitější matematické formulace sázíme mimo plynulý text. 
	Lze umístit několik výrazů na jeden řádek, ale pak je třeba tyto 
	vhodně oddělit, například příkazem \verb$\quad$.\par 
	
	$$\sqrt[x^2]{y_0^3}\quad{N} =\{0,1,2,\ldots\}\quad x^{y^y}\neq x^{yy}
	\quad z_{i_j}\not\equiv z_{ij}$$
	
	V~rovnici (\ref{rov1}) jsou využity tři typy závorek s~různou 
	explicitně definovanou velikostí.\par
	
	\setcounter{equation}{0}
	\begin{eqnarray}
	\label{rov1}
	\left\{\Big[\left(a+b\right)*c\Big]^d+1\right\}&=&x\\
	\lim_{x \rightarrow\infty}\frac{\sin^2x+\cos^2x}{4}&=&y \nonumber
	\end{eqnarray}
	
	
	
	V~této větě vidíme, jak vypadá implicitní vysázení limity 
	$\lim_{n\rightarrow\infty} f(n)$ v~normálním odstavci textu. 
	Podobně je to i s~dalšími symboly jako $\sum_{1}^n$ či $\bigcup_{A\in \mathcal{B}}$. V~případě vzorce $\displaystyle\lim_{x\rightarrow 0}$$\frac{\sin x}{x}=1$ 
	jsme si vynutili méně úspornou sazbu~příkazem \verb$\limits$.\par
	
	\begin{eqnarray}
	\int\limits_a^b f(x)\textup{dx}&=&-\int_b^a f(x)\textup{dx}\\
	\left(\sqrt[5]{x^4}\right)^\prime=\left(x^{\frac{4}{5}}\right)^
	\prime&=&\frac{4}{5}x^{-\frac{1}{5}}=\frac{4}{5\sqrt[5]{x}}\\
	\overline{\overline{A\vee B}}&=&\overline{\overline{A}\wedge\overline{B}}
	\end{eqnarray}

%%-----------------Matrix section----------------------------------		
	\section{Matice}
	
	Pro sázení matic se velmi často používá prostředí array a závorky (\verb$\left$, \verb$\right$).\par
	
	$$\left(
	\begin{array}{cc}
	{a+b} & {b-a}\\
	\widehat{\xi+\omega} & \hat{\pi}\\
	\vec{a} & \overleftrightarrow{AC}\\
	0 & {\beta}\\
	\end{array}
	\right)$$
	
	
	$$\textbf{A}=\left\|
	\begin{array}{cccc}
	a_{11}&a_{12}&\ldots&a_{1n}\\
	a_{21}&a_{22}&\ldots&a_{2n}\\
	\vdots&\vdots&\ddots&\vdots\\
	a_{m1}&a_{m2}&\ldots&a_{mn}\\
	\end{array}
	\right\|$$
	
	$$\left|
	\begin{array}{cc}
	t & u\\
	v~& w
	\end{array}
	\right| =tw - uv $$
	
	Prostředí \verb|array| lze úspěšně využít i jinde.\par
	
	$$\binom{n}{k} = \begin{cases} \frac{n!}{k!(n-k)!} &  
	\textup{pro}\ 0\leq k\leq n \\ 0 &  \textup{pro\ $k<0$\ nebo}\ k>n \end{cases}$$

%%-----------------Conclusion section------------------------------		
	\section{Závěrem}
	
	V~případě, že budete potřebovat vyjádřit matematickou konstrukci 
	nebo symbol a nebude se Vám dařit jej nalézt v~samotném \LaTeX u, 
	doporučuji prostudovat možnosti balíku maker \AmS-\LaTeX.
	Analogická poučka platí obecně pro jakoukoli konstrukci v TeXu.
	
\end{document}