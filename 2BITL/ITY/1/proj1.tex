%%-----------------Document Type----------------------------------
\documentclass[a4paper,11pt, twocolumn]{article}

%%-----------------Packages---------------------------------------
\usepackage[utf8]{inputenc}
\usepackage[czech]{babel}
\usepackage[IL2]{fontenc}
\usepackage{times}
\usepackage[left=2cm, text={17cm,24cm}, top=2.5cm]{geometry}

%%-----------------User commands----------------------------------
\newcommand\quot[1]{\quotedblbase #1\textquotedblleft}

%%-----------------Title------------------------------------------
\title{Typografie a publikování\\ 1. projekt}
\author{Jozef Urbanovský\\ xurban66@stud.fit.vutbr.cz}
\date{}

\begin{document}
\maketitle

%%-----------------1.section--------------------------------------
\section{Hladká sazba}
{Hladká sazba je sazba z~jednoho stupně, druhu a řezu pí­sma sázená na stanovenou šířku odstavce. Skládá se z~odstavců, které obvykle začínají­ zarážkou, ale mohou být sázeny i bez zarážky -- rozhodují­cí­ je celková grafická úprava. Odstavce jsou ukončeny východovou řádkou. Věty nesmějí začínat číslicí.\par
Barevné zvýraznění­, podtrhávání­ slov či různé velikosti písma vybraných slov se zde také nepoužívá. Hladká sazba je určena především pro delší­ texty, jako je napří­klad beletrie. Porušení­ konzistence sazby \-působí v~textu rušivě a~unavuje čtenářův zrak.}

%%-----------------2.section--------------------------------------
\section{Smíšená sazba}
{Smíšená sazba má o~něco volnější­ pravidla než hladká sazba. Nejčastěji se klasická hladká sazba doplňuje o~další řezy pí­sma pro zvýraznění­ důležitých pojmů. Existuje \quot{pravidlo}:
\begin{quotation}Čím více \textbf{druhů, \emph{řezů,}} {\scriptsize velikostí}, barev pí\-sma a jiných efektů použijeme, tím \emph{profe\-sionálněji} bude dokument vypadat. Čtenář tím bude vždy {\huge nadšen!}\end{quotation}\par
\textsc{Tímto pravidlem se \underline{nikdy} nesmíte řídit.}\\ Příliš časté zvýrazňování textových elementů a~změny velikosti {\tiny písma} {\Large jsou} {\LARGE známkou} \textbf{\huge{amatéris\-mu}} autora a~působí \emph{\textbf{velmi} rušivě}. Dobře navrže\-ný dokument nemá obsahovat více než 4 řezy či druhy písma. {\tt Dobře navržený dokument je decentní, ne chaotický.}\par
Důležitým znakem správně vysázeného dokumentu je konzistentní používání různých druhů zvýraznění. To například může znamenat, že \textbf{tučný řez} písma bude vyhrazen pouze pro klíčová slova, \textsl{skloněný řez} pouze pro doposud neznámé pojmy a~nebude se to míchat. Skloněný řez nepůsobí tak rušivě a~používá se častěji. V~\LaTeX u jej sázíme raději příkazem \verb$ \emph{text} $ než \verb$\textit{text}$.\par
Smíšená sazba se nejčastěji používá pro sazbu vědeckých článků a~technických zpráv. U~delších dokumentů vědeckého či technického charakteru je zvykem upozornit čtenáře na význam různých typů zvýraznění v~úvodní kapitole.}

%%-----------------3.section--------------------------------------
\section{České odlišnosti}
{Česká sazba se oproti okolnímu světu v~některých aspektech mírně liší. Jednou z~odlišností je sazba uvozovek. Uvozovky se v~češtině používají převážně pro zobrazení přímé řeči. V~menší míře se používají také pro zvýraznění přezdívek a~ironie. V~češtině se používá tento \textbf{\quot{typ uvozovek}} namísto anglických ``uvozovek''. Lze je sázet připravenými příkazy nebo při použití UTF-8 kódování i přímo.\par
Ve smíšené sazbě se řez uvozovek řídí řezem prvního uvozovaného slova. Pokud je uvozována celá věta, sází se koncová tečka před uvozovku, pokud se uvozuje slovo nebo část věty, sází se tečka za uvozovku.\par
Druhou odlišností je pravidlo pro sázení­ konců řádků. V české sazbě by řádek neměl končit osamocenou jednopí­smennou předložkou nebo spojkou. Spojkou \quot{a} končit může při sazbě do 25 liter. Abychom \LaTeX u zabránili v~sázení osamocených předložek, vkládáme mezi předložku a~slovo \textbf{nezlomitelnou mezeru} znakem \verb$~$ (vlnka, tilda). Pro automatické doplnění vlnek slouží volně šiřitelný program \emph{vlna} od pana Olšáka \footnote{Viz ftp://math.feld.cvut.cz/pub/olsak/vlna/.}.}

\end{document}